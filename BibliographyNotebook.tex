\documentclass[12pt,a4paper]{article}
\usepackage[utf8]{inputenc}
\usepackage[english]{babel}
\usepackage{amsmath}
\usepackage{amsfonts}
\usepackage{amssymb}
\usepackage{makeidx}
\usepackage{graphicx}
\usepackage[left=2cm,right=2cm,top=2cm,bottom=2cm]{geometry}
\usepackage{natbib}
\bibliographystyle{plainnat}
\author{Adrian BACH}
\title{NewBibliography}
\begin{document}

\section*{Template}

\subsubsection*{Title}

\textbf{Authors}: \cite{}.

\textbf{Journal}: .

\textbf{Publication}: .

\textbf{Consulted}: .

\textbf{Keywords}: .

\textbf{Points of interest}: .

\textbf{Critique}:

\newpage

\section*{Management of conservation conflicts}

\subsubsection*{Resource Management Cycles and the Sustainability of Harvested Wildlife Populations}

\textbf{Authors}: \cite{fryxell2010resource}.

\textbf{Journal}: Science Reports.

\textbf{Publication}: May 2010.

\textbf{Consulted}: August 2020.

\textbf{Keywords}: Adaptive management, harvesting effort, harvesting quotas, deterministic model, stability analysis, disturbance.

\textbf{Points of interest}: deterministic model of harvest and growth with and without 'environmental stochasticity' in which effort (number of harvesters) can vary with harvesters sharing information. Stochasticity is modelled by the standard deviation of a normal distribution, probably centred on the population size computed according to harvest and recruitment. Without environmental stochasticity, the population manage to get back to its initial value after the disturbance of harvest season, but within several decades of yearly updates of harvest quota, during which effort and quotas synchronize, both in lag behind the population. With environmental stochasticity, there are no dampened oscillation getting the pop back to initial. Not sure what quotas and effort synchrony means; that they vary together? Effort is immediately linked with quota? Quotas determine the number of harvesters without lag? They also compared the extinction risk between a constant effort (limited number of licences for example) and a dynamic one (anyone can apply for the licence each year and their number varies with quota settings), and it was significantly less likely with the constant than with the dynamic one. Quotas are set yearly, before the hunting season; they recommend to reassess population more often and set quotas more regularly accordingly. They used their model to predict three feature of the system and tested it against three actual cases.
SEVERAL INTERESTING REFERENCES.

\textbf{Critiques}: Oscillations might be dampened because of the strong hypothesis that managers are reluctant to set a quota higher than a previous maximum.
 
\newpage

\subsubsection*{Conflicting interests of ecosystem services: Multi-criteria modelling and indirect evaluation of trade-offs between monetary and non-monetary measures}

\textbf{Authors}: \cite{wam2016conflicting}.

\textbf{Journal}: Ecosystem services.

\textbf{Publication}: 2016.

\textbf{Consulted}: June 2021.

\textbf{Keywords}: Trade-off modelling, land multi-use, optimizing model, deterministic, maximum expected value, land sharing.

\textbf{Points of interest}: A very dense trade-off model with interactive leslie matrices for a boreal forest (logging), moose population (game hunting), cattle and sheep population (livestock grazing). Each activity yields an expected value in euros. Optimization of the land productive value according to different scenarios: optimizing for wood production, optimizing for game hunting, optimizing for livestock grazing, optimizing for the total net value without constraints on activities, with low level constraints on cattle sheep and moose populations (a bit of each), and finally with high level constraints on them (a lot of each). Assumed that the land is owned by a logging stakeholder. Conclusion: the most profitable option for all stakeholders is for hunters and sheep herders to give away a part of their expected gains for every one, included them, to get the most of the land productive capacity. Interestingly, with such a complex system, the Pareto optimum (the point where no stakeholder can improve their situation without hindering the others'), considered the best possible option by economists, is not the optimal situation here because in this situation, cattle herders and moose hunter are much worse off than the constrained multi-use scenarios, where every stekeholder can make the most of their activity.

\textbf{Critique}: I don't see where the conservation question is here, since all the natural features involved in the model are grown for production. No uncertainty included. No within group equity assessment. No dynamic behaviour of stakeholders groups.

\newpage

\subsubsection*{Reallocating budgets among ongoing and emerging conservation projects}

\textbf{Authors}: \cite{wu2020reallocating}.

\textbf{Journal}: Society for Conservation Biology.

\textbf{Publication}: Accepted July 2020.

\textbf{Consulted}: July 2020.

\textbf{Keywords}: budget allocation optimization, adaptive management, Return-On-Investment, conservation optimization.

\textbf{Points of interest}: Comparison of the general outcomes of conservation at a national scale between 3 strategies: (S1) no budget allocation to omitted or new projects, (S2) reallocate budget only from terminated projects, (S3) reallocate dynamically according to projects ROI. Comparison according to the frequency of new project appearance, and to the frequency of ROI reassessment. Two case study for simulation: invasive weed species take over risk, and bird extinction risk in Australia. In the weeds case, S2 and S3 were slightly more efficient than S1, especially for frequent ROI reassessment and frequent appearance of new threats. But for birds, there was no big difference between S1 and S3, supposedly because the appearance of a new threatened bird species is not frequent, but I don't see much argument to this.

\newpage

\subsubsection*{A multispecies assessment of wildlife impacts on local community livelihoods}

\textbf{Authors}: \cite{pozo2020multispecies}.

\textbf{Journal}: Conservation biology.

\textbf{Publication}: August 2020.

\textbf{Consulted}: August 2020.

\textbf{Keywords}: Multi-species system, conservation conflicts, incident assessment.

\textbf{Points of interest}: Different species can affect human livelihood, and affect it differently. The occurrence of incidents varies across the year : Elephants and hippopotamuses raid more crops during the wet season, predators kill cattle more frequently during dry season. Interesting to think about it when modelling the system into GMSE.

\newpage

\section*{Timing of intervention}

\subsubsection*{How Much Information on Population Biology Is Needed to Manage Introduced Species?}

\textbf{Authors}: \cite{simberloff2003information}.

\textbf{Journal}: Conservation Biology.

\textbf{Publication}: February 2003.

\textbf{Consulted}: September 2020.

\textbf{Keywords}: Timing of investment, waiting, biological knowledge, invasive species management, eradication.

\textbf{Points of interest}: The author argues that waiting to enhance ecological, biological knowledge on the invasive species is not always (not to say rarely) a good solution to manage alien species invasion. Among the studies produced, only a few will be effectively useful for management, and in the mean time impacts on local ecosystems get increasingly important. The authors recommends acting as soon as possible, preferably eradicating the alien species before it settles, while scientific studies are carried out.

\textbf{Critique}: I find the author position too aggressive towards alien species. 'Alien' invasion of new ecosystems has always been part of population dynamics, and is, for the most part, caused by our way of life (importations, exportations, travel, migrations, ...). It is quite unfair for us to let alien species bear the costs of this. Of course, local populations need to be protected and maintained, but this should not mean the complete extermination of an alien species. Gaining scientific knowledge on the implication of such invasions also allows to find softer ways to deal with it. In my opinion, managers should not rush in implementing drastic policies of eradication without having figured out other softer possibilities. The introduction of new species in an environment is a natural process in population dynamics, and we should not prevent it in my opinion. Models are a useful tool to rapidly investigate scenarios and test hypotheses or strategies, it would be a waste not to use them before eradicating an animal population. Moreover, invasions can now involve several species already interacting (cats and rats for example) and as they arrive together, eradicate them asynchroneously can have very detrimental impacts on local species. And come on! We're not going to kill everything that doesn't look local in order to protect what we assumed was here first right?!

\newpage

\section*{Population dynamics}

\subsection*{Competition theory}

\subsubsection*{How to quantify competitive ability}

\textbf{Authors}: \cite{halt2018quantify}.

\textbf{Journal}: Journal of Ecology.

\textbf{Publication}: 02/2018.

\textbf{Consulted}: 07/2022.

\textbf{Keywords}: intra/inter-specific competition, competition factor, experimental design for competition ability.

\textbf{Points of interest}: In a 2-species competition system (Beverton-Holt equation), the dominant species is the one that has both the highest growth rate in absence of competition and the strongest tolerance to intra and inter-specific competition. See Appendix S1 for different version of this equation. A measure of competition ability to predict which would win. Depends on growth rate in absence of competitor, per capita response to conspecifics density and per capita response to competitor density. Decomposition of intra-specific competition into RESPONSE and EFFECT. Response is the variation of growth rate in presence of conspecific(s), effect is the change implied by the focal individual presence to conspecific'(s) growth rate. BUT effect does not seem to play a role in ability to resist competition.

\textbf{Critique}: No mention of apparent competition. No acknowledgement that the value of these parameters might vary greatly between conspecifics, which would potentially mean many replication of the same experiment for each species.

\newpage

\subsection*{Apparent competition}

\subsubsection*{SIMPLE RULES FOR INTER-SPECIFIC DOMINANCE IN SYSTEMS WITH EXPLOITATIVE AND APPARENT COMPETITION}

\textbf{Authors}: \cite{holt1994simple}.

\textbf{Journal}: The American Naturalist.

\textbf{Publication}: 1994.

\textbf{Consulted}: April 2021.

\textbf{Keywords}: Apparent competition, resource competition, 1 predator 2 preys 1 basal resource, deterministic ODE model, stability analysis, invasion analysis.

\textbf{Points of interest}: A deterministic ODE model of apparent competition, with the two preys competing also for resources. When resource competition only, the prey that resists invasion is the one for which the amount of resource consumed at equilibrium is the lowest (R* rule), because the winning species will exhibit a higher growth rate for the same consumption. When apparent competition only, the prey that maintain is the one that tolerate the highest density of predator at equilibrium (P* rule), because the winning prey withstand a predator density that causes a negative growth rate on the other prey. The paper investigates what's happening when the two systems are combined. If the predation effort is evenly distributed on both preys, the one that maintains is the one that both depletes the resources by the lowest and tolerates the highest predation (both rules hold R** and P**). But it can vary with resource availability, if prey 2 is more efficient than prey 1 in assimilating resources when they are low and conversely, then prey 2 will invade when low resources availability, and prey 1 when high availability, but P** and R** rules still stand. Even when one species is better at exploiting resources while the other is better at withstanding predation. Which prey maintains then depend on the resource pool size, but P** and R** rules still stand. When predation is asymmetric (MBC's intercept depends on the conversion rate of the predator), there is a panel of four possible outcomes where which rule stands and which does not depends on initial conditions (which prey is preferred by predator -- MBCs relative positions -- how good it withstands predation -- prey isoclines slopes -- and which is the best resource competitor -- position of prey isoclines intercepts.)

\textbf{Critique}: I don't understand the role density-independent mortality in the system. 

\newpage

\subsubsection*{Cats protecting birds: modelling the mesopredator release effect}

\textbf{Authors}: \cite{courchamp1999cats}.

\textbf{Journal}: Journal of animal ecology.

\textbf{Publication}: December 2001.

\textbf{Consulted}: August 2020.

\textbf{Keywords}: Hyperpredation, insular alien species, logistic growth system, stability analysis, simultaneous control.

\textbf{Points of interest}: A deterministic model of logistic growth and trophic interaction between introduced cat, introduced rabbit, and indigenous bird species. Stability analysis shows that bird population is maintained in two scenarios (under specific conditions): cats management only and the three species persist; simultaneous control of cats and rabbits and only birds remain. Theoretical evidence of hyperpredation apparent competition mechanism. They set cats K as the number of available preys over their predation rate at time t, meaning it's a moving K value, interesting! They mention that a timing analysis could be interesting!
LOTS OF APPARENT COMPETITION CASES REFERENCES.

\textbf{Critiques}: They set cats preference towards the indigenous prey because they are less adapted and able to defend, but I think it is a strong hypothesis that could potentially change their result. After 20 years of non-management, at the start of the control scheme, bird population goes very close to 0, which is highly unwanted because Allee effect or stochastic environmental change could drive the population to extinction regardless of trophic interaction (unless allee effects are not possible in the equation?).

\newpage

\subsubsection*{Golden eagles, feral pigs, and insular carnivores: How exotic species turn native predators into prey}

\textbf{Authors}: \cite{roemer2002eagle}.

\textbf{Journal}: PNAS.

\textbf{Publication}: January 2002.

\textbf{Consulted}: May 2020.

\textbf{Keywords}: Population dynamics model, Lotka-Volterra, apparent competition, hyperpredation.

\textbf{Points of interest}: Lotka-Volterra model of four species, apex predator eagle preying on the three other, two competing predators Fox and Skunk for smaller prey, an abundant eagle prey ferral piglets. Without piglets, eagle cannot settle and nest, and the endemic fox win the competition with skunk. Introduction of pig allow the eagle to settle and grow in numbers, intensifying predation on local fox and skunks. Since fox is more diurnal than skunk, predation on foxes is more likely and they suffer predation harder than skunks.
Interesting ways of testing the robustness of their model: varying the parameter they estimated and measure the variance in the results. Results were a lot more sensible to piglet-related parameters than any other. Cool ways of estimating the parameters as well.

\newpage

\subsubsection*{Removing Protected Populations to Save Endangered Species}

\textbf{Authors}: \cite{courchamp2003removing}.

\textbf{Journal}: Science.

\textbf{Publication}: November 2003.

\textbf{Consulted}: May 2020.

\textbf{Keywords}: multi-species Lotka-Voltera, apparent competition, eagle fox pig, conservation challenge.

\textbf{Points of interest}: Removal of ferral pigs that enabled eagle settlement is expected to lead to fox extinction through hyperpredation. With current conditions (after removal of most of the eagles) it can lead to a significant increase but through a pit after the first year, during which stochastic dynamics threaten foxes with a high extinction probability. All this could be avoided by simultaneously removing pigs and eagles. But removing the remaining eagles will most likely involve destructive methods, albeit the eagle is protected. So... fox or eagle?

\newpage 

\subsubsection*{Endangered, apparently: the role of apparent competition in endangered species conservation}

\textbf{Authors}: \cite{decesare2010endangered}.

\textbf{Journal}: Animal Conservation.

\textbf{Publication}: July 2010.

\textbf{Consulted}: May 2020.

\textbf{Keywords}: Apparent competition, fitness ratio, exploitation competition, predation competition, predation rate, drivers, review.

\textbf{Points of interest}: Apparent competition is when a prey species induce a decrease of another because they share the same predator species. It can lead to extinction of one of them when the primary prey species have a greater fitness than the secondary one. Because the primary species can maintain under predation and induce the increase of the predator species, if it increases, the predation effort on the secondary species increases as well. And because the secondary one is less incline to maintain under predation pressure, it can eventually lead to its extinction. When the prey species foraging ranges overlap, exploitation competition can add complexity to the system. Potential symptoms of a problematic apparent competition: average fitness ratio of the prey species, niche overlap and predation overlap, predation behaviour (generalist, selective?) and mobility, abundance of primary prey with respect to the secondary, regulatory (predation rate increases with prey density) or depensatory (predation decreases with prey density) predation rate.

\newpage

\subsubsection*{Conservation Strategies for Species Affected by Apparent Competition}

\textbf{Authors}: \cite{wittmer2012conservation}.

\textbf{Journal}: Society for conservation biology.

\textbf{Publication}: January 2013.

\textbf{Consulted}: June 2020.

\textbf{Keywords}: Apparent competition, comparative study, management strategies.

\textbf{Points of interest}: Comparative case studies of 4 management strategies of endangered populations under apparent competition: (1) control predator population, (2) control primary prey population, (3) control both, and (4) not intervene.
(1) Predator control usually lead to an strong increase in primary prey population, leading to stop control, after what the prey available are numerous and predator population increases even higher than before, intensifying predation on the endangered species.
(2) If the predator is generalist, primary prey control increases predation on the alternative prey which is highly unwanted, but can be efficient by reducing predator population when its specific.
(3) Simultaneous control of primary prey and predators yielded the best results, but struggled to get social approval.
(4) Not intervening when the apparent competition is human induced usually leads to a slow but sure decline in the secondary prey population.
Two studies involving models.
\newpage

\subsubsection*{Mind the cat: Conservation management of a protected dominant scavenger indirectly affects an endangered apex predator}

\textbf{Authors}: \cite{KROFEL201640}.

\textbf{Journal}: Biological Conservation.

\textbf{Publication}: March 2016.

\textbf{Consulted}: May 2020.

\textbf{Keywords}: Apparent competition, bear-lynx interaction, kleptoparasitism, bear management.

\textbf{Points of interest}: As bear density increased thanks to human-maintained feeding sites, the frequency of bear encounter with lynx kill sites increased. Bear feed on lynx killings, while lynx usually feed on them for several days. Increased killing effort for lower time feeding on the prey is decreasing lynx population which is threatened. Need to place the feeding sites further from lynx predation range.

\newpage

\subsubsection*{Apparent Competition, Lion Predation, and Managed Livestock Grazing: Can Conservation Value Be Enhanced?}

\textbf{Authors}: \cite{ngweno2019apparent}.

\textbf{Journal}: Frontiers in ecology and evolution.

\textbf{Publication}: April 2019.

\textbf{Consulted}: May 2020.

\textbf{Keywords}: Apparent competition, lion conservation, ungulate predation, density dependant growth rate, glades.

\textbf{Points of interest}:
Markers of predator-mediated apparent competition on secondary prey populations: selective killing according to Jacob's index; Allee effect (negative growth rate at low or high densities) in presence of predator and primary preys but regular growth without the predator (growth rate decreases with density); survival decreases with proximity to primary prey presence zones.
Jacob's index: from -1 (highly avoided) to 1 (highly selected), and 0 if killing according to density.
The primary prey's index is close to zero, so predator kill them relatively to its abundance. The secondary prey is closer to 1, so predator kill them when they find them. Even when primary preys are abundant, if a secondary prey appears, the predator will preferably kill them.
Zebra are attracted to glades because they are nutrient-rich, so they gather there and are more susceptible to be killed. But if the glades are close to the secondary preys life zones, they are also more likely to be predated. But when the primary prey population regulates well under predation, the secondary one undertake an Allee effect that can lead to extinction.
A better planning of glades locations regarding the secondary prey life zones is a management solution, alternative to go back to culling lions.

\newpage

\subsubsection*{Do rabbits eat voles? Apparent competition, habitat heterogeneity and large-scale coexistence under mink predation}

\textbf{Authors}: \cite{oliver2009rabbits}.

\textbf{Journal}: Ecology Letters.

\textbf{Publication}: 2009.

\textbf{Consulted}: April 2021.

\textbf{Keywords}: Apparent competition, model selection, prey habitat connectivity, field experiment.

\textbf{Points of interest}: Apparent competition system with mink preying on two species: rabbits on grass and voles on the river borders, meaning non-overlapping habitats. Check the microcosm experiment-based assumption that habitat connectivity heterogeneity can maintain a apparent competitive system (both prey persist over time) in large-scale landscape. Assess several covariates relative to habitat characterization + distance matrix between grass land (rabbits) and river banks (voles). Model selection with AICc. Estimation of the predation range of minks based on the distance matrix by log likelihood. Same experiment in a zone nearby were mink is absent to be more confident in the fact that the effect observe id linked to mink presence. When present, voles potential habitats well connected to grasslands are empty, and around 40\% are occupied when badly connected to grassland. Habitats occupancy increase with connectivity when mink is absent. Conclusion, managers should consider connectivity between the preys' pools when conserving. It is also an important aspect to consider when modelling spatially in GMSE (should the population share all the landscape or niches?).

\textbf{Critique}: Interpretation of the p-value beyond acceptance or rejection of the null hypothesis (they interpret the value as the importance of the variable in the variance explained, the AICc between two models with and without the variable could tell that, or the weight of the variable in a PCA of the covariates).

\newpage

\section*{IBMs}

\subsubsection*{Individual-based modelling in ecology: what makes the difference?}

\textbf{Authors}: \cite{uchmanski1996ibm}.

\textbf{Journal}: Trends in Ecology and Evolution.

\textbf{Publication}: October 1996.

\textbf{Consulted}: August 2021.

\textbf{Keywords}: Theory, IBMs, criteria, framework.

\textbf{Points of interest}: Definition: Genuine 'individual-based' models describe a population made up of individuals that may differ from one another; they also describe changes in numbers of individuals rather than in the population density, and take resource dynamics explicitly into account. WHEN AND WHY AN INDIVIDUAL-BASED APPROACH IS RELEVANT: four criteria. (1) Life cycle complexity (2) spatially explicit environment (food availability, connectivity, physical conservation measures, etc) (3) Need for counts in natural numbers (4) Variability among individuals of the same age. Can be free from carrying capacity-linked dynamics, and from the assumption of a 'perfect natural equilibrium'. IMPORTANT TO ESTIMATE THE COMPLEXITY NEEDED COMPARED TO MATHEMATICAL MODELS. Mathematical models better for very large populations (bacteria, micro-organisms, large-scale fish populations, etc), IBMs better for small, local-scale, specific populations (mammals, local fish communities, CONSERVATION CASES). Importance of intra-specific competition: IBMs model interactions better. Models better inter-individual variability (resource assimilation, reproduction, hunt, etc) making it ideal for selection-driven evolution mechanisms.

\textbf{Critique}:

\newpage

\section*{Not PhD related}

\subsection*{Evolution theory}

\subsubsection*{Fifty years of the Price equation}

\textbf{Authors}: \cite{lethtonen2020price}.

\textbf{Journal}: Philosophical Transactions of the Royal Society B.

\textbf{Publication}: March 2020.

\textbf{Consulted}: May 2020.

\textbf{Keywords}: Price equation, evolutionary, theory, introduction.

\textbf{Points of interest}: General presentation of the Price equation.
\begin{equation}
\bar{w}.\Delta\bar{g} = cov(w,g) + E[w.\Delta g]
\end{equation}
$w$ individual fitness. $g$ is allele frequency WITHIN AN INDIVIDUAL. $\Delta{g}$ is the variation of $g$ from a generation to the next. Bars are averages across the population. $E[]$ is an average over all the individuals in a generation.
The covariance describes the intensity of natural selection on the allele: 0 neutral, $<0$ deletarious, $>0$ beneficial.
$E[w.\Delta g]$ represents the expected intensity of vertical transmission.

\textbf{Questions}: Did I interpreted covariance right? What is an allele frequency within an individual? The number of copy of the allele in the genome?

\newpage

\subsubsection*{Branching patterns in phylogenies cannot distinguish diversity-dependent diversification from timedependent diversification}

\textbf{Authors}: \cite{pannetier2020branching}.

\textbf{Journal}: Evolution.

\textbf{Publication}: 2020.

\textbf{Consulted}: November 2020.

\textbf{Keywords}: Lineage diversification, maximum likelihood, model selection, time VS diversity, phylogeny.

\textbf{Points of interest}: Lineages often exhibit a slowdown over time, is it linked to an increase in extinction / a decrease in diversification due to niche competition, or simply to a time dependant process (such as slow-pace changing of environmental conditions)? Comparison of a null model of lineage diversification over time only dependant on time, against a model with time-dependence AND diversity-dependence (negative feedback of diversity on diversification, the more diverse the lower the diversification rate). The null model was built such that the expected number of species would appear the same as the diversity dependant model, in order to narrow the potential differences to the prevalence of density dependence only. Simulations for several extinction rates and divergence ages (tree age) for the same carrying capacity (different meaning for each model: when and how strongly does the slowdown occurs for null, and maximum niche capacity for the DD model) and
To avoid false positives, bootstrapping method to show the first 5 and 95 percentiles on the distributions to define a likelihood zone where it is impossible to distinguish which model best fits the data.
When generating random lineage with the models, the two distribution overlap in most parameter combinations.
The main point is that, even if the likelihood of the model with diversity dependence fits the data, it is impossible to tell if it's because of it or because of other time-dependent processes without testing the null model as well.
Example with 5 well known lineages with strong suspicion of density-dependant processes, only one falls outside the zone, meaning that these processes do not leave a clear trace to be detected.

\textbf{Critiques}: Isn't niche competition a time-dependent process itself? Or is it to fine-grained compared to evolution time scale?\\
Would it be relevant to test a model where diversity has a positive feedback on extinction rate?\\
Where is the stochasticity in the simulations?

\newpage

\subsection*{Ethology}

\subsubsection*{A theoretical exploration of dietary collective medication in social insects}

\textbf{Authors}: \cite{POISSONNIER201878}.

\textbf{Journal}: Journal of Insect Physiology.

\textbf{Publication}: August 2017.

\textbf{Consulted}: May 2020.

\textbf{Keywords}: Agent-based modelling, collective immune response, social diet change, pathogen spread, nutritional geometry.

\textbf{Points of interest}: In a hypothetical insect colony, individual can change their intake target to the immunity response one when interacting with a infected nest mate. If the change is long (long immune response), the collective immune response is very efficient but costly to uninfected workers because the intake target they switched to does not maximizes their fitness. For a similar probability of engaging in a social immune response, it is much less efficient when the change is short (short immune response). But varying the probability of engaging in a social immune response does not affect the efficiency of the long immune response, whereas increasing the probability of engaging lead the short immune response to the same efficiency as the long one, while uninfected workers keep their optimal intake target for longer.
When infected, foragers have a hindered foraging capacity traduced by a probability to fail bring food back to the hive, which is comparable to a quarantine for infected. When pathogen spread in the colony is low, the probability of failing does not have much effect on the collective immune response. But for fast spreading pathogen, high failing probability lead the colony to lack nutrient intake and thus being unable to engage an efficient immune response.

\textbf{Critique}: Is the individual based approach really justified? 

\newpage

\subsubsection*{Ant Foragers Compensate for the Nutritional Deficiencies in the Colony}

\textbf{Authors}: \cite{csata2020ant}.

\textbf{Journal}: Current Biology.

\textbf{Publication}: January 2020.

\textbf{Consulted}: May 2020.

\textbf{Keywords}: Collective choice, nutritional challenges, nutrient deprivation compensation, pheromone trail, agent-based model.

\textbf{Points of interest}: Through individual decision of eating or not eating on a food source according to their nutritional state, foragers were able to collectively compensate for the nutrients their colony were deprived from. Ant flow, probability of feeding were assessed on a 1-hour two-choices (food source containing or not containing the missing nutrient) experimental set up. In average, the colonies chose the food containing the missing nutrient, even when it was as fine as a single essential amino acid, and even when it was mixed with sugar. An individual model hypothesizing that ants choose the path with the highest pheromone concentration, lay a fixed amount of pheromones when they move, but add an extra when they are loaded was able to fit the experimental results, suggesting that no more than these mechanisms are needed for foragers to balance the colony nutrient intakes. 

\newpage

\subsection*{Complex systems stability}

\textbf{Authors}: \cite{duthie2020component}.

\textbf{Journal}: Nature Scientific Reports.

\textbf{Publication}: May 2020.

\textbf{Consulted}: May 2020.

\textbf{Keywords}: Complex systems, stability analysis, response variance.

\textbf{Points of interest}: Variance in the system's different components response to perturbation increases the potential for system's stability.

\newpage

\newpage
\bibliography{PhD-allReadings.bib}
\nocite{*}

\end{document}